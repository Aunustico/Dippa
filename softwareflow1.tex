% Define block styles
\tikzstyle{decision} = [diamond, draw,, aspect=2, fill=blue!20, 
    text width=7.5em, text badly centered, node distance=2.5cm, inner sep=0pt]  
\tikzstyle{block} = [rectangle, draw, fill=blue!20, 
    text width=5em, text centered, node distance=2.5cm, rounded corners, minimum height=4em]   
\tikzstyle{invisible} = [inner sep=0,minimum size=0, node distance = 2.5cm]
\tikzstyle{line} = [draw, -latex']
\tikzstyle{cloud} = [draw, ellipse,fill=red!20, node distance=2.5cm,
    minimum height=2em]
    
\begin{figure}
\centering
\begin{tikzpicture}[node distance = 2.0cm, auto]
    \node [block] (INIT) {Initialize};
    \node [cloud, left of = INIT, node distance = 4cm] (START) {Start};
    \node [decision, below of = INIT] (INTERRUPT) {Interrupt};
    \node [block, right of = INTERRUPT, node distance = 5cm] (READACC) {Read acceleration data};
    \node [decision, below of = READACC] (SAVEDATA) {Save data};
    \node [block, below of=SAVEDATA] (SAVETOSD) {Save data to SD-card};
    \node [invisible, left of = SAVEDATA] (MID2) {};
    \node [invisible, left of = SAVEDATA, node distance = 5cm] (MID1) {};
    \node [invisible, left of = SAVETOSD, node distance = 5cm] (MID6) {};
    \node [invisible, below of = MID6, node distance = 1.5cm] (MID10) {};
    \node [decision, below of = MID10, node distance = 1.5cm] (TIMER) {Timer};
    \node [invisible, below of = TIMER, node distance =1.5 cm] (MID7) {};
    \node [decision, below of = MID7, node distance = 1.5 cm] (TIMEOUT) {Timeout};
    \node [invisible, below of = TIMEOUT, node distance = 1.5 cm] (MID8) {};
    \node [decision, below of = MID8, node distance = 1.5 cm] (ERRORS) {Errors};
    \node [invisible, below of = ERRORS, node distance = 1.5cm] (MID9) {};
    \node [decision, below of = MID9, node distance = 1.5cm] (BUFFER) {Buffer empty}; 
    \node [block, right of = TIMER, node distance = 5 cm] (READTEMP) {Read Temperature};
    \node [block, right of = TIMEOUT, node distance = 5 cm] (SETFLAG) {Set interrupt flag};
    \node [block, right of = BUFFER, node distance = 5 cm] (WRITEDATA) {Write buffer to SD-card};
    \node [block, right of = ERRORS, node distance = 5 cm] (HANDLE) {Handle errors}; 
    \node [invisible, left of = BUFFER, node distance = 4cm] (MID4) {};
    \node [invisible, left of = INTERRUPT, node distance = 4cm] (MID5) {};
    \node [invisible, below of = WRITEDATA, node distance = 1.75] (MID11){};
    \node [invisible, below of = MID4, node distance = 1.5 cm] (MID12){};
    \path [line] (INIT) -- (INTERRUPT);
    \path [line] (START) -- (INIT);
    \path [line] (INTERRUPT) -- node {true}(READACC);
    \path [line] (READACC) -- (SAVEDATA);
    \path [line] (SAVEDATA) -- node {true}(SAVETOSD);
    \path [draw] (SAVEDATA) -- node[above] {false} (MID2);
    \path [line] (MID2) -- (MID1);
    \path [line] (SAVETOSD) |- (MID10); 
    \path [draw] (INTERRUPT) --  node[left]{false}(MID1);
    \path [line] (MID1) -- (TIMER);   
    \path [draw] (TIMER) -- node [left]{false} (MID7);
    \path [line] (MID7) -- (TIMEOUT);
    \path [draw] (TIMEOUT) -- node [left]{false} (MID8);
    \path [line] (MID8) -- (ERRORS);
    \path [draw] (ERRORS) -- node [left]{false} (MID9);
    \path [line] (MID9) -- (BUFFER);   
    \path [line] (TIMER) -- node[above] {true} (READTEMP);
    \path [line] (READTEMP) |- (MID7);
    \path [line] (TIMEOUT) -- node{true} (SETFLAG);
    \path [line] (SETFLAG) |- (MID8);   
    \path [line] (BUFFER) -- node {false} (WRITEDATA);
    \path [line] (ERRORS) -- node {true} (HANDLE);
    \path [line] (HANDLE) |- (MID9);
    \path [line] (BUFFER) -- node[above] {true}(MID4);
    \path [draw] (MID4) -- (MID5);
	\path [line] (MID5) -- (INTERRUPT);    
    \path [draw] (WRITEDATA) |- (MID12);
    \path [draw] (MID12) -- (MID4) ; 
\end{tikzpicture}
\caption{The software flow of the first data recording software. The rounded rectangles represents tasks whereas the diamonds are conditions. Each conditions are followed by branches depending on the current situation. The tasks and conditions forms a repetitive loop of executions excluding the initialize task.}
\label{softwareflow1}
\end{figure}