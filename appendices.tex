%
%\section{Esimerkki liitteestä\label{LiiteA}}
%
%Liitteet eivät ole opinnäytteen kannalta välttämättömiä ja 
%opinnäytteen tekijän on 
%kirjoittamaan ryhtyessään hyvä ajatella pärjäävänsä ilman liitteitä.
%Kokemattomat kirjoittajat, jotka ovat huolissaan
%tekstiosan pituudesta, paisuttavat turhan 
%helposti liitteitä pitääkseen tekstiosan pituuden annetuissa rajoissa.
%Tällä tavalla ei synny hyvää opinnäytettä.   
%
%Liite on itsenäinen kokonaisuus, vaikka se täydentääkin tekstiosaa.
%Liite ei siten ole pelkkä listaus, kuva tai taulukko, vaan 
%liitteessä selitetään aina sisällön laatu ja tarkoitus. 
%
%Liitteeseen voi laittaa esimerkiksi listauksia. Alla on 
%listausesimerkki tämän liitteen luomisesta. 
%
%%% Verbatim-ympäristö ei muotoile tai tavuta tekstiä. Fontti on monospace.
%%% Verbatim-ympäristön sisällä annettuja komentoja ei LaTeX käsittele. 
%%% Vasta \end{verbatim}-komennon jälkeen jatketaan käsittelyä.
%\begin{verbatim}
%	\clearpage
%	\appendix
%	\addcontentsline{toc}{section}{Liite A}
%	\section*{Liite A}
%	...
%	\thispagestyle{empty}
%	...
%	tekstiä
%	...
%	\clearpage
%\end{verbatim}
%
%Kaavojen numerointi muodostaa liitteissä oman kokonaisuutensa:
%\begin{eqnarray}
%d \wedge A  &=& F, \label{liitekaava1}\\
%d \wedge F  &=& 0. \label{liitekaava2}
%\end{eqnarray}
%
%
%\clearpage
%\section{Toinen esimerkki liitteestä\label{LiiteB}}
%
%%% Liitteiden kaavat, taulukot ja kuvat numeroidaan omana kokonaisuutenaan
%
%Liitteissä voi myös olla kuvia, jotka
%eivät sovi leipätekstin joukkoon:
%%% Ympäristön figure parametrit htb pakottavat
%%% kuvan tähän, eikä LaTeX yritä siirrellä niitä
%%% hyväksi katsomaansa paikkaan. 
%%% Ympäristöä center voi käyttää \centering-
%%% komennon sijaan
%%%
%\begin{figure}[htb]
%\begin{center}
%\includegraphics[height=8cm]{kuva2}
%\end{center}
%\caption{Kuvateksti, jossa on liitteen numerointi}
%\label{liitekuva}
%\end{figure}
%%%
%Liitteiden taulukoiden numerointi on kuvien ja kaavojen kaltainen:
%\begin{table}[htb]
%\caption{Taulukon kuvateksti.}
%\label{liitetaulukko}
%\begin{center}
%\fbox{
%\begin{tabular}{lp{0.5\linewidth}}
%9.00--9.55  & Käytettävyystestauksen tiedotustilaisuus (osanottajat
%ovat saaneet sähköpostitse valmistautumistehtävät, joten tiedotustilaisuus
%voidaan pitää lyhyenä).\\
%9.55--10.00 & Testausalueelle siirtyminen
%\end{tabular}}
%\end{center}
%\end{table}
%Kaavojen numerointi muodostaa liitteissä oman kokonaisuutensa:
%\begin{eqnarray}
%T_{ik} &=& -p g_{ik} + w u_i u_k + \tau_{ik},  \label{liitekaava3} \\
%n_i    &=& n u_i + v_i.                      \label{liitekaava4}
%\end{eqnarray}