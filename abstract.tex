
%% Suomenkielinen tiivistelmä
%% Kaikki tiivistelmässä tarvittava tieto (nimesi, työnnimi, jne.) käytetään
%% niin kuin se on yllä määritelty.
%% Tiivistelmän avainsanat
%%
\keywords{Kiimantarkkailu, Lypsylehmä, \\ Kiihtyvyysanturi}
%% Tiivistelmän tekstiosa
\begin{abstractpage}[finnish]
Tässä opinnäytetyössä tutkitaan sensoreihin perustuvia lypsylehmän kiimantunnistus menetelmiä. Tutkimusta varten kerättiin kiihtyvyys- ja lämpöanturidataa lehmiltä kaulalle kiinnitetyillä datan keruulaitteilla. Työn lopputuloksena on kolme erilaista kiihtyvyysanturin dataan perustuvaa algoritmia. Kaikki algoritmit suoriutuivat kiimantunnistuksesta. Kuitenkin passiivisuuden tunnistukseen perustuva algoritmi osoittautui kaikista luotettavimmaksi ja varmimmaksi erilaisilla parametreilla. Työn johtopäätöksenä on, että lypsylehmän kiima on tunnistettavissa kiihtyvyysanturiin perustuvalla sensorilaitteella. Kuitenkin varmojen johtopäätösten tekemiseksi, tämän työn tulokset tulisi vielä vahvistaa onnistuneilla siemennyksillä.
\end{abstractpage}

%% Pakotetaan uusi sivu varmuuden vuoksi, jotta 
%% mahdollinen suomenkielinen ja englanninkielinen tiivistelmä
%% eivät tule vahingossakaan samalle sivulle
%%
\newpage
%
%% Opinnäytteen ostikko englanniksi. Poista, jos et tarvitse sitä.
\thesistitle{Sensor Based Dairy Cow Estrus Detection}
%\supervisor{Prof.\ Pirjo Professori}
%\advisor{D.Sc.\ (Tech.) Olli Ohjaaja}
\advisor{M.Sc. (tech.)\ Samuli Mäkinen}
\degreeprogram{Electronics and electrical engineering}
\department{Department of Automation and Systems Technology}
\professorship{Autonomous Systems}
%% Abstract keywords
\keywords{Estrus Detection, Dairy Cow,\\ Accelerometer}
%% Abstract text
\begin{abstractpage}[english]
This research studies sensor based dairy cow estrus detection. For the study, we recorded motion and temperature data with a collar sensor. The data was used in algorithm development end evaluation. In result, we developed three different algorithms, all suitable for micro-controller based devices. All the developed algorithms succeeded in the estrus detection against a reference system. However, inactivity detection based algorithm was the most reliable and tolerant to different configurations. In conclusion, the estrus is detectable with accelerometer based sensors. However, in order to make secure conclusions, the results of this study should be verified by a successful insemination. 
\end{abstractpage}

%%% Force new page so that the Swedish abstract starts from a new page
%\newpage
%%
%%% Ruotsinkiellinen tiivitelmä. Poista, jos et tarvitse sitä.
%%% 
%%% Opinnäytteen ostikko ruotsiksi.
%\thesistitle{Arbetets titel}
%%\supervisor{Prof.\ Pirjo Professori}
%\advisor{TkD\ Olli Ohjaaja} %
%%\advisor{M.Sc.\ Tina Tutkija}
%\degreeprogram{Elektronik och elektroteknik}
%\department{Institutionen för radiovetenskap och -teknik}%
%\professorship{Kretsteori}  %
%%% Abstract keywords
%\keywords{Nyckelord p\aa{} svenska,\\ Temperatur}
%%% Abstract text
%\begin{abstractpage}[swedish]
% Sammandrag p\aa{} svenska.
% Try to keep the abstract short, approximately 
% 100 words should be enough. Abstract explains your research topic, 
% the methods you have used, and the results you obtained.  
%\end{abstractpage}

%% Note that if you are writting your master's thesis in English, place
%% the English abstract first followed by the possible Finnish abstract